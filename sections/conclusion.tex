Starting with unrealistically low number of bugs, the initial model was generalized to accommodate for a finite, arbitrary $u$ bugs by discretizing testing time into intervals and the number of failures per interval was found to be multinomially distributed. Equipped with said information, it is necessary for businesses to derive (un)conditional reliability measures for the software by the means of calculating the expected number of failures up to a time $t$ given that $m_e$ bugs had already occurred or not. The theory was then applied to the data in Appendix \ref{table-failure} to find the Maximum Likelihood Estimators for the model's parameters. These results were applied to the situation in MathWorks to determine the release date of StatWorks.

However, the model is not without faults. From the beginning it was assumed that bugs are independent (one bug does not cause another) and any bug can be fixed without the introduction of new bugs. This is rarely true in practice because one bug can cause another and to fix such a bug, an unidentified number of other bugs must be fixed first which can be time-consuming. Furthermore, while fix a given bug it is very well possible that the repair teams introduced a new bug (or bugs) instead. The possibility that the detection time per bug follows a distribution other than exponential was neglected, a fact which may have restricted our options and findings.